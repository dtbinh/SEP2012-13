\Chapter {System Features}
\section {Manual Control}
\subsection {Description and Priority}

The operator can control movement and speed over the robot. The operator needs to hold a
directional button on the interface to move and release it to stop. To change speed, the operator can select the
appropriate speed on the slider bar. It is essential that the functions are implemented in the system to
move the robot to the starting point safely.

The manual control implements requirement R0001, R0002, R0003, R0010

\subsection {Stimulus/Response Sequences}

The operator has to select manual mode to enter manual control. The buttons in the
`Control' tab is then highlighted and is ready to use. The operator selects the appropriate speed by
moving the slider bar before holding on a directional button. The host software will send a PC packet
with the latest speed setting and the direction to the robot, moving the robot in the
indicated direction.

\subsection(Functional Requirements)

The functional requirements of the manual control are as follows:
\begin{itemize}
	\item The GUI includes a `Control' tab that has directional buttons of moving
	forward, moving backwards, rotating left and rotating right.
	\item The `Control' tab also has a sliding bar that toggles the speed setting of 0,1,2,3 and 4. A speed
	setting of 0 halts the movement of the robot while the speed setting of 4 set the fastest possible
	speed for the robot.
	\item If the robot enters a no-go zone , an emergency mechanism is triggered.
\end{itemize}

%The robot can enter external object even if under manual control%

\section {Automatic Control}
\subsection {Description and Priority}
The host software comes with an automatic navigation algorithm. The robot has a light sensor to detect a mine
object and an ultrasonic sensor for detecting an obstacle object. Each time a wall is encountered by
the robot, the robot sends a request to the host software controller to record the position and decide on the best possible path to take. 
The new generated path is then reflected on the GUI while the controller sends back the new path in a PC packet to the robot.

The navigation system may start with either a new, partial or completed map. In any case, the map
will contain information for the starting position of the robot,above-ground walls and buried walls

The automatic path Finding navigation implements requirements  R0003, R0004, R0005, R0006,
R0007, R0009, R0011.
\subsection {Stimulus/Response Sequences}

When the robot set on the starting position of the survey area , the operator has the option to switch to auto
mode. This function disables manual controls while enabling the usage of the start and stop buttons.
At this point, the operator can press the start the button which causes the host software to initialize the
robot with the starting position coordinates and an intial generated path. The robot then follows this
path till it encounters an object, making change to map and requesting a new path.

\subsection {Functional Requirements}

The GUI contains a start and a stop button for auto mode. If the user push a start button, a map is to be generated and be displayed on screen.


\section{Graphical User Interface}
\subsection {Description and Priority}
The GUI serves as an interaction platform between the operator, the robot and the survey area.
. All controls are made available to the operator through the GUI. The GUI also includes executing
either automatic or manual robot control. In addition, the GUI provides a visual interpreta-
tion of the area to the operator in the form of a writable XML map. As the robot
makes its exploration, any objects encountered by the robot is displayed on the GUI. Lastly, the robot's status such as
 oordinates, mode, etc is shown at the bottom of the GUI.

The graphical user interface implements requirements R0007, R0008, R0009, R0010, R0011

\subsection{Stimulus/Response Sequences}

There are various components of the GUI that the operator can access:

\begin{itemize}
	\item The tool bar on top of the GUI - The operator has the ability to load map, save map and save as
	map through the `File' header. If the `Robot' header is accessed, the operator could connect and disconnect
	the robot. Under the `View' header, the grid and path taken by the robot can be toggled on and
	off. As for the last header named `System', the operator has the option to exit the system.
	\item The map interface and map editor - The map interface displays a loaded XML map illustrates the survey area. A map editor is provided for the operator to
	add no-go zones and remove no-go zones. 
	\item Manual mode - When the operator selects manual mode, he has access to the manual controls of
	the robot under the `Control' tab. Manual controls include moving the robot and altering the speed
	settings. 
	\item  Auto mode - Auto mode is used when the robot is placed at a designated start position in the
	survey area. When the operator selects auto mode, the manual functions are disabled. If the start button is
	pressed, the robot will commence autonomous mapping. At any time the operator can click
	on the stop button to halt the operation. 
	\item Robot status -The status displays the robot's coordinates, the current
	mode, the speed of movement, connection and battery strength.
\end{itemize}
\subsection{Functional Requirements}
\begin{itemize}
	\item Directional buttons and speed toggle are needed for the manual mode.
	\item A start and a stop button are required for the auto mode.
	\item A mode switch is required.
	\item The GUI must support the functions of loading and saving a map.
	\item No-Go zones can be edited.
	\item The map must display the wall if the robot discovers one.
	\item The robot status must be displayed.
\end{itemize}

\section{Use Cases}
\subsection{Frequent Use Case}
\paragraph{UC001: Operating the robot in manual mode}
UC001 is associated with section 4.1 - Manual control as well as R0001.
\begin{enumerate}
	\item Use Case Name: Operating the robot in manual mode.
	\begin{itemize}	
		\item Description: The use case shows the operation of the robot under complete manipulation of the operator.
		\item Goal: The objective of this use case is to ensure that the robot under the control of the
	\end{itemize}
	\item Flow of Events:The robot is placed in an arbitrary position outside of the survey area while
	the operator is controlling the robot through Bluetooth.
	\begin{itemize}
		\item The operator switched the robot on and run the program. He then
	executes the GUI, connecting the robot. Next, he selects the manual mode.
		\item The control buttons are highlighted to imply that it is ready for use.
	The operator selects the appropriate speed by using the slider bar.
		\item The operator then holds down the directional key to move the robot till it reaches the destination		
	\end{itemize}
	\item Preconditions:
	\begin{itemize}
		\item	The robot batteries are charged and the robot is working properly.
		\item	The GUI must be working
		\item	The Bluetooth connection operates normally.
		\item	An appropriate XML formatted map must be loaded. This map may be a new, partial or
completed map.
	\item	Postconditions:
	\begin{itemize}
		\item The robot reaches the destination safely.
	\end {itemize}
\end{enumerate}
\paragraph {UC002: Operating the robot in auto mode}
UC002 is associated with section 4.2 - Automatic control as well as R0003.
\begin{enumerate}
	\item Use Case Name: Operating the robot in automatic mode.
	\begin{itemize}
		\item Description: Under the supervision of the operator, the robot commences autonomous mapping from
		a given starting point of the suvery area.
		\item Goal: The end result of this use case is achieved when the robot has successfully mapped all the points in the area.
	\end{itemize}
	\item  Flow of Events: Using manual controls, the robot is moved onto the starting position of the
survey area. Like UC001, the operator supervises the robot through the host machine.
	\begin{itemize}
		\item The operator ensures switched on the robot,executing the program. He then
opens the host machine to run the GUI. The operator connects the robot through Bluetooth, selecting the automatic mode.
		\item The start and stop buttons are then ready to use.
		\item The operator then presses the start button to initiate autonomous search.
		\item Without being interupted, the robot moves on its own till it finishes mapping. The actions are now determinied by the controller in the host machine.
		\item If the operator wishes to halt the progress of the autonomous search, he has to press the stop
button. At any point, the map is written into an XML File and can be loaded again for later
use.
	\end{itemize}
	\item Preconditions:
	\begin{itemize}
		\item The robot batteries are charged and the robot is working properly.
		\item The GUI on the host machine must be running.
		\item The connection between robot and host machine is stable
		\item An appropriate XML formatted map can be loaded. This map may be a new, partial or
completed map.
	\end{itemize}
	\item Postconditions:
	\begin{itemize}
		\item The robot finished mapping completely and safely.
	\end{itemize}
\end{enumerate}
\paragraph{UC003: The operator specifies no-go zone}
\subsection Exceptional Use Cases
\paragraph {UC004: Loss of communication signal}
\begin{enumerate}
	\item Use Case Name:Loss of communication signal.
	\begin{itemize}
		\item Descripton: The process details on the steps taken by the robot in the event if the Bluetooth
communication link is broken between the host software and the robot.
		\item Goal: The robot tries to re-establish communication link.
	\end{itemize}
	\item Flow of Events:
	\begin{itemize}
		\item During manual or autonomous mode, communication between host software and
robot is lost.
		\item The robot stops at its current position.
		\item At the same time, it tries to re-establish communication by prompting the host software
at a regular interval.
		\item If communication is established, the robot will request the next instruction.
		\item Otherwise, it will remain at its current position.
	\end{itemize}
	\item Preconditions:
	\begin{itemize}
		\item The robot is having a connection with the host software.
	\end{itemize}
\item Postconditions:
	\begin{itemize}
		\item The robot should stop immediately
		\item The connection should be resumed after a short time.
	\end{itemize}
\end{enumerate}