%
%Author: Jun Chen 1206265
% 
%

\documentclass[12pt, a4paper]{report}
\usepackage{graphicx}
\usepackage{fullpage}
\usepackage{url}
\usepackage{epstopdf}
\usepackage{listings}
\usepackage{xcolor} 
\pagestyle{headings}

%%% page parameters
\headsep = 25pt
\begin{document}
\oddsidemargin -0.5 cm
\evensidemargin -0.5 cm
\textwidth 15 cm
\topmargin -1.2 cm
\textheight 25 cm
\begin{center}
\includegraphics[scale=1.5]{./UniLogo}\\[3cm]    
\textbf{\Huge \bfseries Software Engineering \\ and Project}\\[2.5cm]



% Title
\textbf{\huge Archaeology Robot}\\[0.5cm]
\textbf{ \huge Test Report }\\[2cm]



\begin{tabular}{ |c | p{2cm} |}
	\hline
Jun Chen 1206265 & \\[.5cm] \hline
\end{tabular}


\vfill

% Bottom of the page
{\large \today}

\end{center}


\tableofcontents


\clearpage


\chapter{Introduction}
\label{cha:I}
In this test plan, I am doing some tests about the map structure. The tests I make will test the functionalities of map. The functionalities include:
\begin{enumerate}
	\item  Find any pixels in the map.
	\item  set a NoGoZone Circle.
	\item  get the length of the map.
	\item  get the width of the map.
	\item  set the length of the map.
	\item  set the width of the map.
	\item  get the ID of the map.
	\item  set the ID of the map.
	\item  get the Start xPos.
	\item  get the Start yPos.
	\item  set the Start xPos.
	\item  set the Start yPos.
	\item  get the current pixel.
	\item  get the start pixel.
	\item  get the goal pixel.
	\item  check two maps are the same or not.
	\item  create a clone of the map.	
\end{enumerate}
There are many methods in the Map class, each of them represents a function of the map. The tests will ensure all the methods is working properly.


\chapter{Test Cases} % (fold)
\label{cha:T}

\section{Testing Environment}
\textbf{Testing tool: } \\
Eclipse SDK for Java Developers: Indigo Version: 3.7.0
\\with JUnit 3.11\\
\textbf{Java version: }\\
Java(TM) SE Runtime Environment (build 1.7.0 07-b11)\\
\textbf{Operating System: }\\
Windows 7\\


\newcounter{casecounter}
\addtocounter{casecounter}{1}
\subsection{Test \arabic{casecounter}}
\paragraph{Test Case Name}: findPixeltest
\paragraph{Type}: JUnit Test
\paragraph{Rationale}:findPixel is used to find a specifical pixel on the map. Its the basic method of this map, so we should ensure it works properly.
\paragraph{Test Objective}: Test the findPixel method of the map. The map should be able to find the specifical pixel on the map.
\paragraph{Test method}: 
\begin{enumerate}
	\item Create a new map.
	\item Find a pixel on the map and then store it into p1. 
	\item Check the xPos and yPos of p1, see if its value is equal to the one map finds.
\end{enumerate}
\paragraph{Pass or Fail}: Pass.

\addtocounter{casecounter}{1}
\subsection{Test \arabic{casecounter}}
\paragraph{Test Case Name}: setNoGoZoneCircletest
\paragraph{Type}: JUnit Test
\paragraph{Rationale}:set No-go zone is a requirement of this project so we need to test it to ensure it works.
\paragraph{Test Objective}: Test and figure out the function pass or not.
\paragraph{Test method}: 
\begin{enumerate}
	\item Create a new map.
	\item set a no-go zone.
	\item set a pixel in the no-go zone
	\item check the pixel's value. It should contain no-go zone value if it is in no-go zone.
\end{enumerate}
\paragraph{Pass or Fail}: Pass.

\addtocounter{casecounter}{1}
\subsection{Test \arabic{casecounter}}
\paragraph{Test Case Name}: getLengthtest
\paragraph{Type}: JUnit Test
\paragraph{Rationale}:Testing getLengthtest method is useful for other methods which will use this method.
\paragraph{Test Objective}: Test and figure out the function pass or not.
\paragraph{Test method}: 
\begin{enumerate}
	\item Create a new map.
	\item get length of the map and store it in a variable x.
	\item check the variable value is same as map's length or not
\end{enumerate}
\paragraph{Pass or Fail}: Pass.

\addtocounter{casecounter}{1}
\subsection{Test \arabic{casecounter}}
\paragraph{Test Case Name}: getWidthtest
\paragraph{Type}: JUnit Test
\paragraph{Rationale}:Testing getWidthtest method is useful for other methods which will use this method.
\paragraph{Test Objective}: Test and figure out the function pass or not.
\paragraph{Test method}: 
\begin{enumerate}
	\item Create a new map.
	\item get width of the map and store it in a variable x.
	\item check the variable value is same as map's width or not
\end{enumerate}
\paragraph{Pass or Fail}: Pass.

\addtocounter{casecounter}{1}
\subsection{Test \arabic{casecounter}}
\paragraph{Test Case Name}: setLengthtest
\paragraph{Type}: JUnit Test
\paragraph{Rationale}:Testing setLengthtest method is useful for other methods which will use this method. With setLength method we can create the map with the size we want.
\paragraph{Test Objective}: Test and figure out the function pass or not.
\paragraph{Test method}: 
\begin{enumerate}
	\item Create a new map.
	\item set a new length value of this new map
	\item get length of the map and store it in a variable x.
	\item check the variable value is same as map's new length or not
\end{enumerate}
\paragraph{Pass or Fail}: Pass.

\addtocounter{casecounter}{1}
\subsection{Test \arabic{casecounter}}
\paragraph{Test Case Name}: setWidththtest
\paragraph{Type}: JUnit Test
\paragraph{Rationale}:Testing setWidththtest method is useful for other methods which will use this method. With ssetWidththtest method we can create the map with the size we want.
\paragraph{Test Objective}: Test and figure out the function pass or not.
\paragraph{Test method}: 
\begin{enumerate}
	\item Create a new map.
	\item set a new width value of this new map
	\item get width of the map and store it in a variable x.
	\item check the variable value is same as map's new width or not
\end{enumerate}
\paragraph{Pass or Fail}: Pass.


\addtocounter{casecounter}{1}
\subsection{Test \arabic{casecounter}}
\paragraph{Test Case Name}: getIDtest
\paragraph{Type}: JUnit Test
\paragraph{Rationale}:Testing setWidththtest method is useful for other methods which will use this method. With getID method we can get the map ID.
\paragraph{Test Objective}: Test and figure out the function pass or not.
\paragraph{Test method}: 
\begin{enumerate}
	\item Create a new map.
	\item get ID of the map and store it in a variable x.
	\item check the variable value is same as map's ID or not
\end{enumerate}
\paragraph{Pass or Fail}: Pass.

\addtocounter{casecounter}{1}
\subsection{Test \arabic{casecounter}}
\paragraph{Test Case Name}: setIDtest
\paragraph{Type}: JUnit Test
\paragraph{Rationale}:Testing setWidththtest method is useful for other methods which will use this method. With setID method we can set the map ID when we call the method.
\paragraph{Test Objective}: Test and figure out the function pass or not.
\paragraph{Test method}: 
\begin{enumerate}
	\item Create a new map.
	\item set a new ID for this map.
	\item get new ID of the map and store it in a variable x.
	\item check the variable value is same as map's new ID or not
\end{enumerate}
\paragraph{Pass or Fail}: Pass.

The other test cases for getXXX methods and setXXX methods are same as above. They test the functionalities of the map.

\addtocounter{casecounter}{1}
\subsection{Test \arabic{casecounter}}
\paragraph{Test Case Name}: compareTotest
\paragraph{Type}: JUnit Test
\paragraph{Rationale}:compareTo method is used to check if two maps are same or not.
\paragraph{Test Objective}: Test and figure out the function pass or not.
\paragraph{Test method}: 
\begin{enumerate}
	\item Create three new maps, two of them are same(map1, map2), one of them is different(map3).
	\item compare map1 to map2 and then store the boolean value into b1.
	\item compare map1 to map3 and then store the boolean value into b2.
	\item check the variable b1 and b2 value is true or false. 
\end{enumerate}
\paragraph{Pass or Fail}: Pass.

\addtocounter{casecounter}{1}
\subsection{Test \arabic{casecounter}}
\paragraph{Test Case Name}: clonetest
\paragraph{Type}: JUnit Test
\paragraph{Rationale}:clone method is used to make a copy of a map.
\paragraph{Test Objective}: Test and figure out the function pass or not.
\paragraph{Test method}: 
\begin{enumerate}
	\item Create a new map.
	\item call clone method and store the new map to map2.
	\item compare map to map2 and then store the boolean value into b.
	\item check the variable b value is true or false. 
\end{enumerate}
\paragraph{Pass or Fail}: Pass.


\newpage
\appendix
\chapter{JUnit Test Cases} % (fold)
\label{cha:JTC}

% chapter junit_test_cases (end)
\begin{lstlisting} 
/**
 * MapStructure Map class Unit Test
 * author Jun Chen (1206265)
 */
package Tests;

import static org.junit.Assert.*;
import java.util.ArrayList;
import MapStructure.Map;
import MapStructure.Pixel;
import MapStructure.RobotStatus;
import org.junit.Test;
 
public class MapStructureTest {
	
	Map map;
//	Pixel p1 = new Pixel(5, 9, 0, map);
	
@Test	
public void findPixeltest(){
	Map map	= new MapStructure.Map("MapNo1", 10, 10, 0, 0, 1);
	Pixel p1 = map.findPixel(5, 9);
	int x = p1.getxPos();
	int y = p1.getyPos();
	assertTrue(x==5);
	assertTrue(y==9);
}

//@Test
//public void setNoGoZonetest(){
//	Map map	= new MapStructure.Map("MapNo1", 10, 10, 0, 0, 1);
//	map.setNoGoZone(1, 5, 5, 5);
//	Pixel p = map.findPixel(2, 3); //it should be in the no-go zone
//	Pixel p2 = map.findPixel(4, 4); //it should be in the no-go zone
//	Pixel p3 = map.findPixel(6, 3); //it should not in the no-go zone
//	int x = p.getValue();
//	int y = p2.getValue();
//	int z = p3.getValue();
//	assertTrue(x == Integer.MAX_VALUE);
//	assertTrue(y == Integer.MAX_VALUE);
//	assertTrue(z == 0 || z == 1 || z == 2 || z == 999 || z == 500);
//}

//@Test
//public void setNoGoZonetest2(){
//	Map map	= new MapStructure.Map("MapNo1", 10, 10, 0, 0, 1);
//	map.setNoGoZone(1, 5, 5, 5);
//	Pixel p = map.findPixel(5, 5); // in the no-go zone (border)
//	Pixel p2 = map.findPixel(1, 1); // in the no-go zone (border)
//	Pixel p3 = map.findPixel(1, 5); //in the no-go zone (border)
//	Pixel p4 = map.findPixel(5, 1); // in the no-go zone (border)
//	int x = p.getValue();
//	int y = p2.getValue();
//	int z = p3.getValue();
//	int r = p4.getValue();
//	assertTrue(x == Integer.MAX_VALUE);
//	assertTrue(y == Integer.MAX_VALUE);
//	assertTrue(z == Integer.MAX_VALUE);
//	assertTrue(r == Integer.MAX_VALUE);
//}
	
@Test
public void setNoGoZoneCircletest(){
	Map map	= new MapStructure.Map("MapNo1", 10, 10, 0, 0, 1);
	map.setNoGoZoneCircle(5, 5, 2);
	Pixel p = map.findPixel(6, 6); //it should be in the no-go zone
	Pixel p2 = map.findPixel(4, 4); //it should be in the no-go zone
	int x = p.getValue();
	int y = p2.getValue();
	assertTrue(x == Integer.MAX_VALUE);
	assertTrue(y == Integer.MAX_VALUE);
}

@Test
public void getLengthtest(){
	Map map	= new MapStructure.Map("MapNo1", 10, 10, 0, 0, 1);
	int x = map.getLength();
	assertTrue(x == 10);
}

@Test
public void getWidthtest(){
	Map map	= new MapStructure.Map("MapNo1", 10, 10, 0, 0, 1);
	int x = map.getWidth();
	assertTrue(x == 10);
}

@Test
public void setLengthtest(){
	Map map	= new MapStructure.Map("MapNo1", 10, 10, 0, 0, 1);
	map.setLength(8);
	int x = map.getLength();
	assertTrue(x == 8);
}

@Test
public void setWidthtest(){
	Map map	= new MapStructure.Map("MapNo1", 10, 10, 0, 0, 1);
	map.setWidth(8);
	int x = map.getWidth();
	assertTrue(x == 8);
}
	
@Test
public void getIDtest(){
	Map map	= new MapStructure.Map("MapNo1", 10, 10, 0, 0, 1);
	String x = map.getId();
	assertTrue(x == "MapNo1");
}

@Test
public void setIDtest(){
	Map map	= new MapStructure.Map("MapNo1", 10, 10, 0, 0, 1);
	map.setId("Maplol");
	String x = map.getId();
	assertTrue(x == "Maplol");
}

@Test
public void getStartxPostest(){
	Map map	= new MapStructure.Map("MapNo1", 10, 10, 0, 0, 1);
	int x = map.getStartxPos();
	assertTrue(x == 0);
}

@Test
public void getStartyPostest(){
	Map map	= new MapStructure.Map("MapNo1", 10, 10, 0, 0, 1);
	int x = map.getStartyPos();
	assertTrue(x == 0);
}

@Test
public void setStartxPostest(){
	Map map	= new MapStructure.Map("MapNo1", 10, 10, 0, 0, 1);
	map.setStartxPos(1);
	int x = map.getStartxPos();
	assertTrue(x == 1);
}

@Test
public void setStartyPostest(){
	Map map	= new MapStructure.Map("MapNo1", 10, 10, 0, 0, 1);
	map.setStartyPos(2);
	int x = map.getStartyPos();
	assertTrue(x == 2);
}

//@Test
//public void getUnexplorePixelstest(){
//	Map map	= new MapStructure.Map("MapNo1", 10, 10, 0, 0, 1);
//	int x = map.getUnexplorePixels();
//	assertTrue(x == 0);
//}

@Test
public void getCurrentPixeltest(){
	Map map	= new MapStructure.Map("MapNo1", 10, 10, 0, 0, 1);
	Pixel p = new Pixel(5, 10, 0, map);
	map.setCurrentPixel(p);
	Pixel p2 = map.getCurrentPixel();
	int x = p2.getxPos();
	int y = p2.getyPos();
	assertTrue(x == 5);
	assertTrue(y == 10);
}

@Test
public void getStartPixeltest(){
	Map map	= new MapStructure.Map("MapNo1", 10, 10, 0, 0, 1);
	Pixel p = new Pixel(1, 1, 0, map);
	map.setStartPixel(p);
	Pixel p2 = map.getStartPixel();
	int x = p2.getxPos();
	int y = p2.getyPos();
	assertTrue(x == 1);
	assertTrue(y == 1);
}

@Test
public void getGoaltest(){
	Map map	= new MapStructure.Map("MapNo1", 10, 10, 0, 0, 1);
	Pixel p = new Pixel(10, 10, 0, map);
	map.setGoal(p);
	Pixel p2 = map.getGoal();
	int x = p2.getxPos();
	int y = p2.getyPos();
	assertTrue(x == 10);
	assertTrue(y == 10);
}

//@Test
//public void outofBoundstest(){
//	Map map	= new MapStructure.Map("MapNo1", 10, 10, 0, 0, 1);
//	Pixel p = new Pixel(15, 18, 0, map);
//	Pixel p2 = new Pixel(2, 2, 0,map);
//	Pixel p3 = new Pixel();
//	boolean out = map.outOfBounds(p);
//	boolean out2 = map.outOfBounds(p2);
//	boolean out3 = map.outOfBounds(p3);
//	assertTrue(out);
//	assertFalse(out2);
//	assertFalse(out3);
//}

@Test
public void compareTotest(){
	Map map	= new MapStructure.Map("MapNo1", 10, 10, 0, 0, 1);
	Map map2 = new MapStructure.Map("MapNo1", 10, 10, 0, 0, 1);
	Map map3 = new MapStructure.Map("MapNo3", 8, 8, 0, 0, 1);
	boolean b1 = map.compareTo(map2);
	boolean b2 = map.compareTo(map3);
	assertTrue(b1);
	assertFalse(b2);
}

@Test
public void clonetest(){
	Map map	= new MapStructure.Map("MapNo1", 10, 10, 0, 0, 1);
	Map map2 = map.clone();
	boolean b = map.compareTo(map2);
	assertTrue(b);
}
}

	

\end{lstlisting} 


\end{document}
