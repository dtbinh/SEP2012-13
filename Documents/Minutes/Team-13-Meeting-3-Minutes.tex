%%%%%%weekly meeting template, prepared by Michael Sheng.  09/03/2007
\documentclass[11pt, a4paper]{article}
\usepackage{times}
\usepackage{ifthen}
\usepackage{amsmath}
\usepackage{amssymb}
\usepackage{graphicx}
\usepackage{setspace}

%%% page parameters
\oddsidemargin -0.5 cm
\evensidemargin -0.5 cm
\textwidth 15 cm
\topmargin -1.2 cm
\textheight 22 cm

\renewcommand{\baselinestretch}{1.4}\normalsize
\setlength{\parskip}{0pt}


\begin{document} 

%%%mention the no, time, and venue of the meeting
%%%\noindent The {\em first} Software Engineering Group Project weekly meeting will be held in {\bf Room 2052} at {\bf 11.10am on Friday 9 March 2007}.


\vspace*{15pt}

\begin{center}
\huge \bf Minutes of Client Meeting 3
\end{center}


\begin{center}
\Large \bf Group 13
\end{center}

\begin{center}
\Large \bf Monday 20th August 2012
\end{center}

%%%first, nominate a chair for the meeting. We suggest that each member at least has one chance as the chair.
%%%\section*{Chair}
\section*{Chair}
Yufeng Bai (a1600095)

\section*{Secretary}
Yatong Zhou(a1204471)

\section*{Members}
1. Dawei Geng (a1219181)\\
2. Nguyen Khoi (a1187070)\\
3. Yunyao Yao (a1203525)\\
4. Shikai Li (a1214223)\\
5. Jun Chen (a1206265)
\vspace*{10pt}

%%%if some students cannot make the meeting due to some reasons, their names should appear here.
\section*{Apologies}
None

%%%short presentation about the work of previous week or any milestone specified in the course.
\section{Time and Place}
The meeting 3 for the Software Engineering and Project was held in Ingkarni Wardli Building, Room 4.62 at 3:30pm on Monday 20th August 2012.

%%%any schedules for this meeting should go next, each with a separate section.
%%%for example, the first meeting is about requirement elicitation, like the following.
\section{Quorum Announcement}
The chairman announced that a quorum of the group was present, and that the meeting, having been duly convened, was ready to proceed with its topic.

\section{Summary of Previous Meeting}
After the second meeting, the basic movement function of robot has been implement and tested by our group members. The basic GUI appearance has been drawn. Team has finished the SRS document at the moment.


\section{Introduction}

\subsection{Current Project Status}
%%%Collect basic functional requirements of {\em Herbs-Online} from diferent stakeholders including the contractor a
Chair Yufeng Bai did a presentation of the project schedule. Including:
\begin{itemize} 
\item Introduction of SRS
\item How do we designed the SRS 
\item Scope of the project
\item User requirements
\item External interface requirements
\item Other requirements
\end{itemize}

\subsection{Client Feedback}
Client gave some suggestions of the current project:
\begin{itemize}
\item Our development team need to consider the latency problem of the robot, an acceptable
latency is 300ms. 
\item The contrast ratio of GUI need a little improvement, so that it can be used better under strong sunshine. (like using the robot in Barr Smith Lawn)
\item We need to pay more attention on spelling check and error detection of the later documentation. 
\item More international mark need to be used in the GUI design part, replace some of the button
icons and word labels with international mark can improve the quality of the user interface. (The less English, the better look GUI will be)
\end{itemize}

\subsection{Introduction Summary}
\begin{itemize}
\item A advanced requirement have been taken to account: User/Developer Mode change. 
\item The robot's move speed should be limited to a certain level to ensure accuracy and safety. 
\end{itemize}


\section{Demonstrate the Project Gantt Chart}
Our team demonstrated the Project Gantt Chart to the client and the feedback is positive.

\section{Adjournment}
The next meeting is a group meeting and will be held in Ingkarni Wardli Building, Room 4.62 at 3:30pm on Monday 27th August 2012.

%%%finally, specifies time of next meeting
\vspace*{10pt}


\end{document}