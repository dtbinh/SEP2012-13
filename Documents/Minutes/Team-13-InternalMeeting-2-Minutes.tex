%%%%%%weekly meeting template, prepared by Michael Sheng.  09/03/2007
\documentclass[11pt, a4paper]{article}
\usepackage{times}
\usepackage{ifthen}
\usepackage{amsmath}
\usepackage{amssymb}
\usepackage{graphicx}
\usepackage{setspace}

%%% page parameters
\oddsidemargin -0.5 cm
\evensidemargin -0.5 cm
\textwidth 15 cm
\topmargin -1.2 cm
\textheight 22 cm

\renewcommand{\baselinestretch}{1.4}\normalsize
\setlength{\parskip}{0pt}


\begin{document} 

%%%mention the no, time, and venue of the meeting
%%%\noindent The {\em first} Software Engineering Group Project weekly meeting will be held in {\bf Room 2052} at {\bf 11.10am on Friday 9 March 2007}.


\vspace*{15pt}

\begin{center}
\huge \bf Minutes of the Second Internal Meeting
\end{center}


\begin{center}
\Large \bf Group 13
\end{center}

\begin{center}
\Large \bf Monday 9th August 2012
\end{center}

%%%first, nominate a chair for the meeting. We suggest that each member at least has one chance as the chair.
%%%\section*{Chair}
\section*{Chair}
 Dawei Geng (a1219181)

\section*{Secretary}
Yufeng Bai (a1600095)

\section*{Members}
1. Nguyen Khoi (a1187070)\\
2. Yatong Zhou (a1204471)\\
3. Yunyao Yao (a1203525)\\
4. Shikai Li (a1214223)\\
5. Jun Chen (a1206265)\\
6. Yunyao Yao (a1203525)
\vspace*{10pt}

%%%if some students cannot make the meeting due to some reasons, their names should appear here.
\section*{Apologies}
None

%%%short presentation about the work of previous week or any milestone specified in the course.
\section{Time and Place}
The second meeting for the Software Engineering and Project was held in Ingkarni Wardli Building, Room 4.23 at 2:00pm on Friday 24th August 2012.

%%%any schedules for this meeting should go next, each with a separate section.
%%%for example, the first meeting is about requirement elicitation, like the following.
\section{Quorum Announcement}
The chairman announced that a quorum of the group was present, and that the meeting, having been duly convened, was ready to proceed with its topic.

\section{Summary of Previous Meeting}
1. The introduction of SRS.\\
2. Describing the scope of the project. \\
3. Explanation of the Users Requirements, External Requirements and Other Requirements.\\
4. Recording some new requirements on GUI and Movement part according to client meeting content.\\
5. Demonstrating the movements of robot in Manual Control Mode.
 

\section{Group Milestone}

\subsection{discussion}
%%%Collect basic functional requirements of {\em Herbs-Online} from diferent stakeholders including the contractor a
1. We need to rebuild the robot to make sure the robot is more accurate.\\
2. We need to pay more attention on the spelling checking and error detection.\\
3. We need to discuss how to implement the Manual Control Operation on GUI.\\
4. We need to improve some parts of the GUI to make it more humanised.


\subsection{Physical Design}
1. Rebuild the robot.

\subsection{Code Design And Test}
1. Keyboard implements Manual Control operation and implements the Manual Control in GUI.\\
2. Fix the contrast ration of GUI and some other drawbacks according to client's requirements from this week's client meeting.\\
3. Map is able to demonstrate the movement of the robot.

\section{Milestone Report}
1. Fill in the Milestone forms and sign off on it.

\section{Documents}
1. Reviewing the report of SRS.\\
2. Previewing the report of the SPMP 


\section{Project Administration}
1. Using a folder to collect all materials for this project


%%%more issues should make it like the above one.
\section{Requirements Elicitation}
1. the colour of arrow on GUI .\\
2. the two wheels of robot need to stop simultaneously.\\
3. The position of the logging information window.\\
4. Using the international symbols instead of the words.  

\section{Adjournment}
The next meeting is a group meeting and will be held in Ingkarni Wardli Building, Room 4.23 at 3:30pm on Monday 27th August 2012.

%%%finally, specifies time of next meeting
\vspace*{10pt}


\end{document}
