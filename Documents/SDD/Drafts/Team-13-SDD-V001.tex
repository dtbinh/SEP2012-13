
% SEP 2012 Group 13
% Software Design Document (SDD)
%
\documentclass[11pt, a4paper]{report}
\usepackage{graphicx}
\usepackage{fullpage}
\usepackage{url}
\pagestyle{headings}

%%% page parameters
\headsep = 25pt
\begin{document}
\oddsidemargin -0.5 cm
\evensidemargin -0.5 cm
\textwidth 15 cm
\topmargin -1.2 cm
\textheight 25 cm
\begin{center}
\includegraphics[scale=1.5]{./UniLogo}\\[3cm]    
\textbf{\Huge \bfseries Software Engineering \\ and Project}\\[2.5cm]



% Title
\textbf{\huge Archaeology Robot}\\[0.5cm]
\textbf{ \huge Test Report }\\[2cm]



\begin{tabular}{ |c | p{2cm} |}
	\hline
Jun Chen 1206265 & \\[.5cm] \hline
\end{tabular}


\vfill

% Bottom of the page
{\large \today}

\end{center}


\tableofcontents



% Version History %

% IMPORTANT %
% Whenever you make a change to this document you MUST put an entry in below
% Must conform to firstName lastName &  date & discription \\ \hline


\clearpage
\section*{Revision History}
\begin{tabular}{| l | l | l | l | }
\hline
Name      		&	Date        	&	Reason For Changes                  	  	&	Version     	\\ \hline
Dawei Geng      &	03 Sep 2012     &	Template			                  	  	&	0.1 	    	\\ \hline
%Name      		&	Date        	&	Reason For Changes                  	  	&	Version     	\\ \hline
%Name      		&	Date        	&	Reason For Changes                  	  	&	Version     	\\ \hline
%Name      		&	Date        	&	Reason For Changes                  	  	&	Version     	\\ \hline
%Name      		&	Date        	&	Reason For Changes                  	  	&	Version     	\\ \hline
%Name      		&	Date        	&	Reason For Changes                  	  	&	Version     	\\ \hline
%Name      		&	Date        	&	Reason For Changes                  	  	&	Version     	\\ \hline
%Name      		&	Date        	&	Reason For Changes                  	  	&	Version     	\\ \hline
%Name      		&	Date        	&	Reason For Changes                  	  	&	Version     	\\ \hline
%Name      		&	Date        	&	Reason For Changes                  	  	&	Version     	\\ \hline
%Name      		&	Date        	&	Reason For Changes                  	  	&	Version     	\\ \hline
%Name      		&	Date        	&	Reason For Changes                  	  	&	Version     	\\ \hline






\end{tabular}
\clearpage

% Introduction %

\chapter{Introduction}% (fold)
\label{cha:I}
%Provide an overview of the SDD and a description of the scope of the system to be developed.%


\section{Purpose and Scope}
\subsection{Purpose}
%Define the purpose of this document, specify intended readership of the docu- ment.%


\subsection{Scope}
%Identify the context of the system; explain what the proposed system does (and what does not, if necessary); describe the relevant benefits, objectives and goals. The description should be consistent with your SRS.%


\section{References}
%Provide a complete list of all documents that you referenced in SDD.%


\section{Overview}
%Overview the content of the document; explain how the SDD is organized.%

\section{Constraints}
%Briefly describe any restrictions, limitations, and constraints that affect the design and implementation of the system.%


% chapter Introduction (end)
\pagebreak


\chapter{System Overview}% (fold)
\label{cha:SO}
%Briefly introduce the system context and design.%


% chapter System Overview (end)
\pagebreak


\chapter{System Architecture and Components Design}% (fold)
\label{cha:SACD}
%In this section, you should cover the following content, in several subsections.%


\chapter{Architectural Description}
%Describe the architectural design of the whole system. Typically, you should include a block diagram showing the major subsystems and their interconnec- tions.%


\section{Component Decomposition Description}
%A decomposition of the subsystems that summarizes the software components.%


\section{Detailed Components Design Description}
%Give the detailed design for each component. In particular, for each component, you need to provide:
%• Component Identifier: An identifier unique to this component.
%• Purpose: A reference (link) back to the requirement specification (SRS).
%• Function: What does this component do? Describe its functionality.
%• Subordinates: The components used by this component.
%• Dependencies: Constraints placed on this component by other compo- nents.
%• Interfaces: Control and data flow in and out of the component.
%• Data: Description of internal data if there is any.


\section{Architectural Alternatives}
%Discuss briefly other architectures that were considered if any.%


\section{Design Rationale}
%Discuss the rationale on selecting the architecture described in 3.1, including the critical issues and tradeoffs that were considered.%


% chapter System Architecture and Components Design (end)
\pagebreak


\chapter{Data Design}% (fold)
\label{cha:DD1}
%You should cover database description and data structures in this section.%


\section{Database Description}
%Describe briefly the database(s) that is part of the system.%


\section{Data Structures}
%Give the detailed design of the database, i.e., entities and their relationships. You can use either ER model or UML for this purpose.%


% chapter Data Design (end)
\pagebreak

\chapter{Design Details}% (fold)
\label{cha:DD2}
%You shall describe your design by using the following diagrams to reflect all the major requirements that you documented in the SRS.%


\section{Class Diagrams}
%Describe all class diagrams that are considered in the system. Give the details (e.g., attributes, operations) associated with the class.%


\section{State Diagrams}
%Present the state diagrams of objects involved.%


\section{Interaction Diagrams}
%Present the interactions of objects involved.%
%Note: all the diagrams shall be numbered and link to the requirements in%
%the SRS; UML notations should be strictly followed.%


% chapter Design Details (end)
\pagebreak


\chapter{Human Interface Design}% (fold)
\label{cha:HID}

\section{Overview of the User Interface}
%Describe briefly the general functionalities of the system from end users’ per- spective.%


\section{Deatiled Design of the User Interface}
%You have to present your user interface design from the following perspectives:
%• Screen Images: screenshots showing the interface from end users’ perspec- tive.
%• Screen Objects and Actions: discussion of screen objects and the actions associated with those objects.
%• Report Forms: a description of major reports provided by the system, if any.
%Note: all the interface designs have to link to the user requirements and functional requirements in the SRS; all the interface images shall be num- bered.


% chapter Interface Design (end)
\pagebreak




\chapter{Resource Estimates} % (fold)
\label{cha:RE}
%A summary of computer resource estimates required for operating the system.%


% chapter Resource Estimates (end)
\pagebreak

\chapter{Definitions, Acronyms, and Abbreviations} % (fold)
\label{cha:DAA}
%Provide definitions of all terms, acronyms, and abbreviations used in SDD.%


% chapter Definitions, Acronyms, and Abbreviations (end)
\pagebreak
% Appendicies %
\newpage
\appendix

\pagebreak

\chapter{}






\end{document}
