\chapter{System Architecture and Components Design}% (fold)
\label{cha:SACD}
%In this section, you should cover the following content, in several subsections.%


\section{Architectural Description}
%Describe the architectural design of the whole system. Typically, you should include a block diagram showing the major subsystems and their interconnec- tions.%

\section{Component Decomposition Description}
%A decomposition of the subsystems that summarizes the software components.%


\section{Detailed Components Design Description}
%Give the detailed design for each component. In particular, for each component, you need to provide:
%� Component Identifier: An identifier unique to this component.
%� Purpose: A reference (link) back to the requirement specification (SRS).
%� Function: What does this component do? Describe its functionality.
%� Subordinates: The components used by this component.
%� Dependencies: Constraints placed on this component by other compo- nents.
%� Interfaces: Control and data flow in and out of the component.
%� Data: Description of internal data if there is any.
\subsection{Interfaces}
\subsubsection{Client}
\begin{itemize}
\item Component Identifier: C-0
\item Purpose:
\item Function: control the movement of the robot such as moving, turning left, turning right, stop
\item Subordinates:client.BluetoothClient & client.RobotClient.
\item Dependencies: none
\item Interfaces: none
\item Data:The interface stores information related to the Client, %which 
%includes command operation bits, definitions for boolean responses with byte values and
%masks to separate data from commands.
\end{itemize}

\subsubsection{Controller}
\begin{itemize}
\item Component Identifier: C-1
\item Purpose: 
\item Function:
\item Subordinates:
\item Dependencies: host.MasterController
\item Interfaces: none
\item Data:none
\end{itemize}
% %Controller
% %ID C-1 Type Interface Class host
% %Purpose Abstracting the method declaration and implementation for calls made by
% %host.ui.UserInterface classes to the Controller classes.
% %%Function Provides a structure for method calls to any implementation. shall define the
% %parameters to be given and returned by methods that use this inerface
% %Subordinates host.MasterController
% %Dependancies none.
% %Interfaces none.
% %Data none.

\subsubsection{Database}
\begin{itemize}
\item Component Identifier: C-2 
\item Purpose: 
\item Function:creating map, loading map and saving map 
\item Subordinates:database.XMLDatabase
\item Dependencies: none
\item Interfaces: none
\item Data:none
\end{itemize}


% Database
% ID C-2 Type Interface Class host.database
% Purpose Abstracting the method declaration and implementation for the database module
% for outside calls.
% Function Provides a structure for method calls to any implementation of a class implementing
% the Database interface.
% Subordinates database.XMLDatabase
% Dependancies none.
% Interfaces none.
% Data none.
% Map
\subsubsection{Map}
\begin{itemize}
\item Component Identifier:  C-3
\item Purpose: 
\item Function: getting and setting the state of the map, getting the position of the robot
\item Subordinates:database.ArrayMap
\item Dependencies: none
\item Interfaces: none
\item Data: none
\end{itemize}
% Map
% ID C-3 Type Interface Class host.database
% Purpose Abstracting the method declaration and implementation of any map functionality
% to be used by a class implementing the Database interface.
% Function Provides a structure for method calls to any implementation.
% Subordinates database.ArrayMap
% Dependancies none.
% Interfaces none.
% Data none.
\subsubsection{User Interface}
\begin{itemize}
\item Component Identifier: C-4
\item Purpose: 
\item Function: allowing the user to change the map and the database, notifying of found walls
\item Subordinates:host.ui.GUI
\item Dependencies: none
\item Interfaces: none
\item Data:none
\end{itemize}
% UserInterface
% ID C-4 Type Interface Class host.ui
% Purpose Abstracting the method declaration and implementation for the GUI module.
% Function Provides a structure for method calls to any implementation. Calls that will
% be implemented by a GUI class are defined here.
% Subordinates host.ui.GUI
% Dependancies none.
% Interfaces none.
% Data none.
% 19
\subsection{Class Files}
\subsubsection{Host}
\begin{itemize}
\item Component Identifier: C-5 
\item Purpose: initialising the system.
\pargraph{SRS requirement:} touches almost all of the requirements
\item Function:The class creates host.database.XMLDatabase, a host.MasterController,
a client.Client and host.ui.GUI object.
\item Subordinates:host.Controller,host.database.Database, host.ui.UserInterface &
 client.Client.
\item Dependencies: host.MasterController,host.database.XMLDatabase, host.ui.GUI,
 client.SimulationClient & client.BluetoothClient.
\item Interfaces: Controller, Database, User Interface, Client
\item Data: storing the module objects and the flag variables to determine a simulation
or real connection.
\end{itemize}
% ID C-5 Type Class Class
% Purpose This class runs the initial setup of the entire system. Defining the interconnections
% between the each component.
% Function This class creates a host.database.XMLDatabase, a host.MasterController,
% a client.Client and host.ui.GUI object. It will pass the host.database.Database
% and client.Client classes nothing, the host.MasterController the already created
% client.Client and host.database.Database objects. It will create the GUI using
% the host.MasterController object and client.Client before adding the GUI to the
% host.MasterController object.This class also accepts an argument of either nothing,
% where it will setup the system for a real connection to the robot, or �-s� to setup the
% system with a simulated client
% Subordinates host.Controller,host.database.Database, host.ui.UserInterface &
% client.Client.
% Dependancies host.MasterController,host.database.XMLDatabase, host.ui.GUI,
% client.SimulationClient & client.BluetoothClient.
% Interfaces host.Controller,host.database.Database, host.ui.UserInterface &
% client.Client.
% Data Holds the four module objects as data and the flag variable to determine a simulation
% or real connection.

\subsubsection{Master Controller}
\begin{itemize}
\item Component Identifier: 
\item Purpose: to coordinate the operation of the entire system. The operations
include commnunication between the host and the client, path finding,
loading and saving map.
\pargraph{SRS requirement:} R0001: Manual Robot Movement, R0003: Automatic Robot Movement, 
R0004: Automatic Exploration, R0005:Avoiding danger under automatic mode, 
R0009: Robot Mode Change, R0011: Load/Save XML
\item Function: .The Master Controller,
 creates client.BluetoothClient object to communicate to the client. It also saves and load a map
 into a XML file. The class makes call to the AutoController for the next move.
\item Subordinates:host.AutomaticUser, client.BluetoothClient, host.database.Point,
host.database.XMLDatabase, host.database.Map & host.ui.GUI
\item Dependencies: host.Controller
\item Interfaces: Database, Map, Point
\item Data:object detection parameters and threshold; minimum move and rotation distances 
\end{itemize}

% MasterController
% ID C-6 Type Class Class host
% Purpose This class will act as part of the interface between the GUI and the client. It
% will use a client.BluetoothClient object to communicate commands to the client. Each
% command should first have a series of checks, for safety purposes, before calling on the
% client.BluetoothClient object. It will also read and write to a database containing all
% relevant information that the GUI requires. This includes the ability to save and load
% a current minefield map into a saved XML file. The MasterController class will make
% calls to the AutomaticUser class to return the next logical move for automated control.
% SRS requirements U-3, U-4, U-5, U-6, U-7, U-8, S-0, S-1, S-2, S-3, N-2 & N-3.
% Function The MasterController class shall create Client, Database & GUI objects to
% interact with. It will pass the GUI a Database object from which to draw its data from and
% itself as a Host object from which it shall relay user commands. The MasterController
% object will have no main method and only function on methods called by the GUI object.
% Subordinates host.AutomaticUser, client.BluetoothClient, host.database.Point,
% host.database.XMLDatabase, host.database.Map & host.ui.GUI
% Dependancies host.Controller
% Interfaces host.database.Database, host.database.Map, host.database.Point,
% host.ui.UserInterface & client.Client.
% Data sweep Offsets, object detection parameters and thresholds, mine detection parameters
% and thresholds & minimum move and rotation distances.
% Host

\subsubsection{AutomaticUser}
\begin{itemize}
\item Component Identifier: C-7 
\item Purpose: to autonomously exploring the map safely
\pargraph{SRS requirement:} R0004: Automatic Exploration, R0005: Avoiding danger
\item Function: The method will take a Map file and then determine the robot�s current location, bearing and then possible moves. 
It always check if a move is safe before  executing 
\item Subordinates:host.database.Map & client.Client.
\item Dependencies: host.Controller
\item Interfaces: Controller
\item Data: the position of the robot relative to the walls
\end{itemize}
% AutomaticUser
% ID C-7 Type Class Class host
% Purpose This class shall perform all functionality to determine a safe autonomously
% moving robot. It shall ensure SRS requirements U-4, N-2 & N-3.
% Function This calls will take a Map object and use it to determine the robot�s current
% location, bearing and then possible moves. The class will follow the state diagram in
% section 5.11.2 of this document. Whenever a manual movement is going to be executed
% host.MasterController will first implement one of AutoMaticUser�s safety check functions
% to check a safe move is to be executed.
% Subordinates host.database.Map & client.Client.
% Dependancies host.Controller
% Interfaces host.Controller
% Data Movement safety check offset co-ordinates relative to the front left wheel.
\subsubsection{BlueToothClient}
\begin{itemize}
\item Component Identifier: IC-8
\item Purpose: to act as an intermediary between the host and client.
\pargraph{SRS requirement:}R0001: Manual Robot Movement, R0004: Automatic Move Robot,
R0005: Automatic robot exploration, R0011:Load and Save XML Files
\item Function: When a command parameter is parsed into the method, the command is sent to 
client.BluetoothHost.
\item Subordinates:client.BluetoothHost & host.MasterController
\item Dependencies: client.Client.
\item Interfaces: Controller, Client
\item Data: none
\end{itemize}
% BluetoothClient
% ID C-8 Type Class Class client
% Purpose To transmit a signal from the host machine to the client and have a reply signal
% to return to the host machine. SRS requirements N-4 and U-10
% Function This Java class shall create an object to be implemented within a BasicHost
% object to implement robot movement, a sweep method and interrogate the robot for data
% readings.. This class shall follow the client interface and be able to communicate with
% a client.BluetoothHost object via a Bluetooth link. it shall transmit a Bluetooth signal
% from the host machine to the client. When a method is called it will receive a parameter.
% It shall then call the send command, passing both the operator command value, declared
% within the Client interface, and the parameter which the original method call was given.
% The send command will then handle the transmission and receiving of data between the
% BluetoothClient and client.BluetoothHost.
% Subordinates client.BluetoothHost & host.MasterController
% Dependancies client.Client.
% Interfaces host.Controller, client.Client
% Data none.

\subsubsection{Bluetooth Host}
\begin{itemize}
\item Component Identifier: 
\item Purpose: to act as an intermediary between the host and client.
\pargraph{SRS requirement:} R0001: Manual Robot Movement, R0004: Automatic Move Robot,
R0005: Automatic robot exploration, R0011:Load and Save XML Files
\item Function: opening a I/O stream to get input from client.BluetoothClient.
Client.BluetoothHost remains in an infinite loop while waiting for a command.
\item Subordinates:client.RobotClient
\item Dependencies: none
\item Interfaces: Client 
\item Data:none
\end{itemize}


% BluetoothHost
% ID C-9 Type Class Class client
% Purpose To receive command trasmissions from the a client.BluetoothClient object.
% SRS requirements N-4 and U-10.
% Function This class opens a Bluetooth I/O stream to receive input from a
% client.BluetoothClient object over the communications interface. client.BluetoothHost
% will remain within an infinite while loop trying to receive a command from the I/O
% stream.
% Subordinates client.RobotClient.
% Dependancies none.
% Interfaces client.Client
% Data none.

\subsubsection{RobotClient}
\begin{itemize}
\item Component Identifier: C-10
\item Purpose: to move the robot and to record the detected features
\pargraph{SRS requirement:}R0001: Manual Robot Movement, R0004: Automatic Move Robot,
R0005: Automatic robot exploration, R0011:Load and Save XML Files
\item Function: The robot client receives command from a client.BluetoothHost and then perform 
accodirngly. If it detects any features of the area, it will return data to client.BluetoothHost.
\item Subordinates: none
\item Dependencies: client.BluetoothHost
\item Interfaces:  Client
\item Data: the position of the robot, the speed and angles that the lightsensor should rotate
, the collected data
\end{itemize}


% RobotClient
% ID C-10 Type Class Class client
% Purpose To physically move the robot using a client.Client interface. SRS requirements
% U-5, U-7 & S-0.
% Function This class will receive method calls from a client.BluetoothHost object. Its
% methods will instruct the Mindstorm robot to respond to commands given by the host
% machine and return data requested by the host machine.
% Subordinates none.
% Dependancies client.BluetoothHost
% Interfaces client.Client
% Data Variables to tune the dimensions of the robot, speed & angles which the lightsensor
% arm must rotate to collect data at the correct points.
\subsubsection{XMLDatabase}
\begin{itemize}
\item Component Identifier: C-11 
\item Purpose: to be a map database for a host machine. Maps can be loaded and saved by the 
class. 
\pargraph{SRS requirement:} R0011: Load and Save XML File
\item Function: initalizing a  database.Map object and allows the creation of a map through
the database.Map class. The class has method to get the state of the map such as pixel data and
different zones. It is able to set and get the position of the client on the map. The current state
of map can be saved to an XML file. A unfinished or finished map can also be loaded into XMLDatabase. 
\item Subordinates:host.database.xmlParser & host.database.ArrayMap
\item Dependencies: host.database.DataBase & host.database.Map
\item Interfaces: Database and Controller
\item Data: none
\end{itemize}

% XMLDatabase
% ID C-12 Type Class Class host.database
% Purpose Act as an XML map database for a host machine. Maps may be loaded and
% saved through this class. SRS requirements U-3, S-2 & S-3
% Function instantiates a database.Map object and allows the creation of a map through
% the database.Map class. Has methods to retrieve zone and pixel data through the
% database.Zone class. Is able to set the Client�s position on the map with a bearing and
% return the data through a getRobotPosition method. The current state of a map shall be
% saved to an XML file at a given file-path when requested. A previous map may then be
% loaded into the XMLDatase when requested.
% Subordinates host.database.xmlParser & host.database.ArrayMap.
% Dependancies host.database.DataBase & host.database.Map
% Interfaces host.database.DataBase & host.Controller
% Data
\subsubsection{Zone}
\begin{itemize}
\item Component Identifier: C-12
\item Purpose: To store the state of each grid zone into 4 pixels (sub-grids) 
\pargraph{SRS requirement:} R0011: Load and Save XML File
\item Function: initilising the 4 pixels as unexplored. The class can set and get the state
of a pixel, which is respresent as a pre-defined integer. 
\item Subordinates: host.database.Map
\item Dependencies: host.database.Map
\item Interfaces: Map
\item Data:variables storing the states of the zone's four areas.
\end{itemize}

% Zone
% ID C-14 Type Class Class host.database
% Purpose To store the state of each grid zone into 4 pixels (sub-grids) that the
% database.XMLDatbase object shall use to gather data on a database.Map�s current state.
% SRS requirement U-3, S-2, S-3.
% Function This class shall initialise the 4 pixels within the database.Zone object as unexplored,
% the unexplored static final variable shall be stored within the database.Database
% interface, and shall be changed though the database.Database Object with a setPixelState
% method. Accompanying the setPixelState method will be a getPixelState method which
% shall return the current state of a pixel as an integer value. State�s are defined within the
% database.Database interface.
% Subordinates host.database.Map
% Dependancies host.database.Map
% Interfaces host.database.Map
% Data The Zone class contains four variables for the Zone�s four quadrants, each is set to
% a state defined in the host.database.Map interface
\subsubsection{Map}
\begin{itemize}
\item Component Identifier: C-13
\item Purpose: to create a map storing states of small grids in a two dimensional arrays
\pargraph{SRS requirement:} R0011: Load and Save XML File
\item Function:initialising the map to be the size of a grid, which is two by two. As the robot
 explores the map, it calls setBoarder method to increase the size of the map. The state of each
 pixel can be set and accessed through the setPixel and getPixel methods
 
\item Subordinates:host.database.Point & host.database.Zone.
\item Dependencies: host.database.Map
\item Interfaces: Map and Database
\item Data: a two-dimessional array storing Zone objects
\end{itemize}

% ArrayMap
% ID C-15 Type Class Class host.database
% Purpose To store the state of each grid as a database.Zone object within a 2d array of
% Zone objects. SRS requirement U-3, S-2 & S-3.
% Function The map shall be created with a height and width of the rectangular grid space.
% A boarder for the minefield may be created within this rectangular grid space with a
% setBoarder method. The state of each pixel of each zone is accessible through setPixel
% and getPixel methods which shall call the methods within database.Zone to return pixel
% states
% Subordinates host.database.Point & host.database.Zone.
% Dependancies host.database.Map
% Interfaces host.database.Map & host.database.Database
% Data this class contains a 2d-array of Zone objects that are stored with an associated
% state variable defined in the host.database.Map class.
\subsubsection{MapPanel}
\begin{itemize}
\item Component Identifier: C-14 
\item Purpose: to display the graphical respresentation of the map
\pargraph{SRS requirement:} R0007: Map Representation
\item Function: The map is displayed using information from the database. The map shows
the position of the robot and its traveled distance
\item Subordinates: host.database.Map & host.database.Database.
\item Dependencies: host.ui.UserInterface
\item Interfaces: Map and Database
\item Data: position of the robot and how far the robot has travelled.
\end{itemize}
% 
% mapPanel
% ID C-17 Type Class Class host.ui
% Purpose To display the graphical representations of the map
% Function The map will be displayed using information stored in the map database and
% set out in a grid
% Subordinates host.database.Map & host.database.Database.
% Dependancies host.ui.UserInterface.
% Interfaces host.ui.UserInterface.
% Data This class contains current GUI position data. How far the user has zoomed in,
% how far he has moved centre of the map in both x and y co-ordinates.
% 27

\subsubsection{GUI}
\begin{itemize}
\item Component Identifier: C-15 
\item Purpose: an interface in the host machine to control and keep track of the robot
\pargraph{SRS requirement:} R0008: Robot representation,R0009: Robot mode change,
R0007: Map representation
\item Function: manually controlling the robot, entering the autonomous exploring mode, updating
the battery and signal status
\item Subordinates: host.ui.MapPanel, host.ui.MasterController & host.ui.XMLDatabase. 
\item Dependencies: host.ui.UserInterface & host.Controller
\item Interfaces: UserInterface and Controller
\item Data: predefined representation of functionalities on the screen, the current state of the robot.
\end{itemize}

% GUI
% ID C-18 Type Class Class host.ui
% Purpose To provide the user with an interface to communicate with the robot and display
% the information of the minefield
% Function Confirms the location of the mine, updates the current display, gets information
% of the battery status, as well as the signal strength. It also connects to the database
% and the host.
% Subordinates host.ui.MapPanel, host.ui.MasterController & host.ui.XMLDatabase.
% Dependancies host.ui.UserInterface & host.Controller
% Interfaces host.ui.UserInterface & host.Controller
% Data This class must have a list of predefined colour variations it will use to show objects
% on the screen. It must also keep track of what mode it is currently in to determine what
% will show on screen.
\section{Architectural Alternatives}
%Discuss briefly other architectures that were considered if any.%
The other alternative design which is considered is letting the robot do the exploration without
waiting command from the host. The idea is shelved because the robot's resource is not capable
of performing the algorithm. 

%Need more details

% Other architectures are considered in the design of the system. One of them we considered is
% to indicate the logically finding a path on the robot itself. Then the robot accesses into the bordered
% areas of the map and detect the objects(mines, obstacles or beacons)in order to perform safe
% and successful mission. This design is scraped on the basis that the robot does not have enough
% memory or processing capacity to store the data and execute the commands efficiently.In another
% design, bordered areas will be represented by a bitmap rather than polygonal shapes. This will
% have simplified the logic involved with clipping and merging zones together as well as navigating
% through the map. However this is inconvenient to save the current map into an XML format file
% and will require converting between polygonal shapes and bitmap formats which is complex. This
% design has a limit on using a bitmap which means a very large map will not be able to represent
% the details or require a large amount of memory.

\section{Design Rationale}
%Discuss the rationale on selecting the architecture described in 3.1, including the critical issues and tradeoffs that were considered.%
The design of the system is layer architecture. The model is chosen for its support of seperation 
and independece. The development is spread across teams with each team being responsible
for a different part. The use of interface allows smooth intergration without risk of 
 incompatible method calls. If there is change in the interface, only the adjacent layer is affected 
 so change is localized. 

% chapter System Architecture and Components Design (end)
\pagebreak


\chapter{Data Design}% (fold)
\label{cha:DD1}
%You should cover database description and data structures in this section.%

\section{Database Description}
%Describe briefly the database(s) that is part of the system.%
The map is represented in XML format. It is created when the robot starts to explore.The map updates every time when the robot explores an area, detects
a wall or a border of the survey area. The information which is stored in the array is translated
and saved into XML, which can be loaded for later use. The detailed functionalities are
as follow:

\paragraph{Creating a map}

The map is initialised to be a size of a grid, which is two by two.
The size of the map is stored as digital number and the coordinates of the border are stored in an
array list.

\paragraph{Recording the features of the area}

\begin{itemize}
  \item  Clear Area: containing no features and is considered to be safe. The status and coordinate
   is saved if the area is explored.
\item	Wall Area: containing walls.The status and coordinate is saved if the area is explored.
\item  No-go-zone area: the robot is not permitted to enter the area. The status and coordinate is 
entered by the operator.
\end{itemize}

\paragraph{Recording the position of the robot}

The coordinates of current position are stored in an array list and are flagged as visited.

\paragraph{Saving}

The map which is explored so far can be saved.

\pargraph{Loading}

The previous map can be loaded,

% The representation of database of the minefield in the host side is represented in XML format. It
% is created when the map on the host side is displayed. It is updated every time when the robot
% explorers a new area, detects a mine, an obstacle or finds a beacons in the minefield. All these
% status are translated into XML and saved in the XML file which can be reloaded for further use.
% The database includes following contents:


\section{Data Structures}
%Give the detailed design of the database, i.e., entities and their relationships. You can use either ER model or UML for this purpose.%
The underlying data structure of the map database is a two dimensional array. Each element
is the zone (grid)'s information, which is indexed by its coordinattiong in the map. Each zone(grid)
consisted of the information of its 4 sub-grids, which is stored in a two-by-two array
